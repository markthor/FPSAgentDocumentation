%!TEX root = ../preamble.tex

\section{Discussion}
\label{sec:discussion}
The discussion is divided into three parts, respectively discussing the VRC, the AIC and the combination.

\subsection{The visual recognition component}
From the results in figure \ref{fig:vpr-acc}, the CNNs estimating the VPR performed very well. The training examples that the networks failed to predict correctly were either when the target were in between partitions, or behind the weapon overlay, as seen in section \ref{sec:incorrectpredictions} of the appendix. The networks were able to correctly classify images, where we found it difficult to locate the target, as seen in figure \ref{fig:hardprediction} and in section~\ref{sec:vdexamples} of the appendix and the varying lighting did not seem to cause incorrect predictions.

\begin{figure}[H]
	\begin{scriptsize}
		\sffamily
		\def\svgwidth{\textwidth}
		\input{img/hardprediction.pdf_tex}
	\end{scriptsize}
	\caption[Difficult VPR classification example]{The deep CNN estimating the VPR correctly predicts the class(green square) with a confidence of 55.7\%, with only four pixels of the target being visible. This example was not used for training. The feature maps produced from this example are included in section~\ref{sec:featuremaps-appendix} of the appendix.}
	\label{fig:hardprediction}
\end{figure}

The incorrect predictions of images were the target is present are generally cases were the target could be classified as being in either partition, depending on the exact location of the center of mass of the target. Consequently, such an incorrect prediction does not have a significant impact on an AIC using the incorrect classification to shoot and aim, as some of the target is present in the incorrect partition. These incorrections should therefore be viewed as a natural consequence of the vague classification definition, and not a failure of the optimisation of the model.

Optimising the model with gradient descent proved to be relatively trivial, performing adequately with two distinct network topologies, no regularisation and a very unbalanced distribution of classes. From these observations we conclude that the problem of estimating the VPR is well suited for deep learning. The network with 12 layers and the network with 6 layers performed equally well, but the deeper network did take approximately twice as long to train. However, we can not conclude that increasing the depth does not potentially increase performance if the VPR is applied to harder problems, and the observation should therefore not discourage the use of deeper networks to solve this type of problem. The distribution of classes of training examples is noticeably unbalanced, with more than a 100 times as many classes in the least represented class, as in the most represented class, as seen in \ref{tab:enemy-dist}. However, the unbalance does not seem to prevent the network from learning to recognize even the least represented classes. While it seems unlikely that the network can learn to recognize a class from less than 200 training examples, it is possible because the features recognized by the convolutional layer are similar for all the classes, as visualised in figure~\ref{sec:featuremapsvpr}. Hence, the training of less represented classes uses the convolutional filters learned from the more represented classes to optimise the fully connected layers.

The experimentation with training with a smaller volume of data showed that the 2600 examples were insufficient to estimate the VPR to a sufficient accuracy\todo{ref to results}. This suggests that a real world application of the VPR would require a larger amount of labelled data, as we assume that targets in the real world are more visually varied and harder to detect than targets in this game.

The results in the performance paragraph of section \ref{sec:results-angular-representation} shows varying success with estimating the AR. We observe that the topology of the network has an impact on the error of the horizontal angle, vertical angle and distance, especially when trained without visual distortion. The accuracy of target detection is not affected by the topology, which is not surprising, as it is a strictly easier problem than estimating the VPR. That the topology has a significant influence on the results, leads us to believe, that the problem can be solved with a smaller error if the network topology is improved. Increasing the amount of training data might also yield smaller errors.

The error of the deep network estimating the AR with visual distortion is more than twice as large as the counterpart without visual distortion. Recall that visual distortion both includes dynamic lighting, detailed textures and weapon overlay. This raises the question of how much each of these factors contribute to the error of the network. The visualised errors of figure~\ref{fig:aecollection}, shows that the network has a large error on some examples, where the weapon overlay does not even partially cover the target. Consequently, the two other factors complicates the target detection when estimating the AR. However, it is expected that the weapon overlay adds error in both angles and distance, even if the network functioned optimally. A target in the lower right corner covered by the weapon overlay would be impossible to detect for the VRC, and would in the worst case scenario have an absolute error of 1 in both angles with a theoretically optimal VRC.

The results of the network estimating the VPR with visual distortion indicates, that the network estimating the AR theoretically should be able to approximate the position of the target regardless of visual distortion. The problems of estimating the VPR and the AR have some commonalities, as a VPR can be classified differently if the target is shifted a few pixels. Precision in target detection is therefore necessary in both problems, and the error observed in the network estimating the AR with visual distortion indicates that the network does not have the ability to detect the target with such a precision. However, the deep network estimating the AR without visual distortion proves that estimation of angles with a relatively small error is possible with deep learning. Therefore it is assumed, that the inability of the network  to estimate the AR with an adequate precision is due to insufficient tuning of the learning process, such as the volume or distribution of the training data, the topology of the network or hyper parameters of gradient descent. The network trained by Chenyi Chen et al.\cite{chen} is estimating features similar to the AR, and uses far more training data and trains for more iterations, which also indicates that the network presented by this project is far from optimal. The difference in quality of feature maps, visualised in section~\ref{sec:featuremaps} also indicates, that the convolutional layers of the network estimating the AR is not as functional as the convolutional layers of the network estimating the VPR, which solidifies this assumption. From a theoretical point of view, the difference in the quality of feature maps is not surprising, as the negative log-likelihood cost function optimises better than the euclidean loss cost function used for regression, as described and visualised by Xavier Glorot et al.\cite{DBLP:journals/jmlr/GlorotB10}. Training the network estimating the AR with the convolutional layers of a trained network estimating the VPR could possibly decrease the AR error.

\subsection{The action inferring component}



\subsection{The pipeline}
The results in figure~\ref{fig:neat-cnn-comparison} show a significant difference between the reduction in performance from using a VRC with the AR and the VPR. The pipeline using a VRC estimating the AR clearly performs worse, and this observation corresponds with the performance evaluation of the individual VRCs in section~\ref{sec:results-angular-representation}. The estimation of the AR is far too inaccurate for the AIC to successfully infer an adequate action, and has its fitness reduced by 73.2\% when using the VRC. The weapon overlay covering the target reduces performance for both approaches, but this factor does not explain the entire 37.7\% fitness reduction from using a VRC with the VPR. From detailed examination of running the pipeline with a VRC estimating the VPR, we observed that the combination of low TPS and classification uncertainty when the target is in between partitions was a challenge. As the AIC is trained with the ground truths, assuming that the target is within a partition after a number of time steps given a specific AIC output is relatively safe. However, when the estimation of the VPR is inaccurate when the target is in between partitions, this assumption does not hold and can lead to unwanted behaviour. This is amplified by the fact, that the AIC using the VPR associates specific partitions with reloading, penalising inaccurate partitioning classification.

The observations from the pipeline experiments harmonises well with the observations from the VRC experiments, and indicates that the VPR is easier estimated by deep learning than the AR.

Running the agent with the pipeline proved performance intensive, and the overall performance of the agent is definitely reduced by running with only 5 TPS. While it could be interesting to see how well the combination could perform, the pipeline experiments prove that it is possible for the combination to learn to aim and shoot with the VPR.



Generalising the feature representation to represent multiple targets...

Security cameras, interception missile system

Unity as game framework(frame variance)