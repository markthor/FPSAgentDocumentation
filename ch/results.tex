%!TEX root = ../preamble.tex

\section{Results}
\label{sec:results}

\subsection{Convolutional neural network experiments}

\subsubsection{Feature maps}
The feature maps shown on figure \ref{fig:featuremaps} are visualised by scaling the output range of $[0,1]$ of every neuron linearly to the grey scale range of $[0,255]$. The leftmost column of feature maps are from the first convolutional layer, and the rightmost is the input to the fully connected layers. As a convolutional layer takes a depth slice of all the previous feature maps as input, their is no apparent connection between the visualised output of the max pooling layer and the following result of the convolutional layer.

\begin{figure}[H]
	\begin{scriptsize}
		\sffamily
		\def\svgwidth{\textwidth}
		\input{img/featuremaps.pdf_tex}
	\end{scriptsize}
	\caption{A small subset of the feature maps produced from running a training example from the lightened arena through the visual partitioning classification deep convolutional neural network. The feature maps highlight the position of the target.}
	\label{fig:featuremaps}
\end{figure}

%Shallow vs deep
%No light vs light

%%!TEX root = ../preamble.tex

\pgfplotstableread{data/angular-nolight-shallow.dat}{\angularNolightShallow}
\pgfplotstableread{data/angular-nolight-deep.dat}{\angularNolightDeep}

\begin{figure}[H]
\begin{tikzpicture}[scale=1]
	\begin{axis}[
			height=7.5cm,
			%title=Cost over iterations,
			xlabel=Iterations,
			ylabel=Cost,
			ymin = 0,
			ymax = 0.6,
			xmin = -640,
			xmax = 16640,
			restrict x to domain=0:16000,
			xticklabel style={rotate=30},
			minor x tick num=1,
			legend pos=north east,
		]
		\addplot+ [cred, mark=none] table [x={iteration}, y={score}] {\angularNolightShallow};
		\addlegendentry{Shallow network}
		\addplot+ [cblue, mark=none] table [x={iteration}, y={score}] {\angularNolightDeep};
		\addlegendentry{Deep network}
	\end{axis}
\end{tikzpicture}
\caption[Training the VRC using AR without visual distortion]{Mean squared error cost over mini-batch gradient descent iterations using AR without visual distortion}
\label{fig:score-nolight-angular}
\end{figure}
%%!TEX root = ../preamble.tex

\pgfplotstableread{data/angular-light-shallow.dat}{\angularLightShallow}
\pgfplotstableread{data/angular-light-deep.dat}{\angularLightDeep}

\begin{figure}[H]
\begin{tikzpicture}[scale=1]
	\begin{axis}[
			title=Cost over iterations,
			xlabel=Iterations,
			ylabel=Cost,
			ymin=0,
			ymax=0.55,
			xmin = -640,
			xmax = 16640,
			restrict x to domain=0:16000,
			xticklabel style={rotate=30},
			minor x tick num=1,
			legend pos=north east,
		]
		\addplot+ [cred, mark=none] table [x={iteration}, y={score}] {\angularLightShallow};
		\addlegendentry{Shallow network}
		\addplot+ [cblue, mark=none] table [x={iteration}, y={score}] {\angularLightDeep};
		\addlegendentry{Deep network}
	\end{axis}
\end{tikzpicture}
\caption{Mean squared error cost over mini-batch gradient descent iterations using AR with visual distortion}
\label{fig:score-light-angular}
\end{figure}


%%!TEX root = ../preamble.tex

\pgfplotstableread{data/nolight-vpr-shallow.dat}{\nolightVPRShallow}
\pgfplotstableread{data/nolight-vpr-deep.dat}{\nolightVPRDeep}

\begin{figure}[h]
\begin{tikzpicture}[scale=1]
	\begin{axis}[
			height=7.5cm,
%			title=Cost over iterations,
			xlabel=Iterations,
			ylabel=Cost,
			ymin = 0,
			xmin = -480,
			xmax = 12480,
			restrict x to domain=0:12000,
			xticklabel style={rotate=30},
			legend pos=north east,
		]
		\addplot+ [cred, mark=none] table [x={iteration}, y={score}] {\nolightVPRShallow};
		\addlegendentry{Shallow network}
		\addplot+ [cblue, mark=none] table [x={iteration}, y={score}] {\nolightVPRDeep};
		\addlegendentry{Deep network}
	\end{axis}
\end{tikzpicture}
\caption[Training the VRC using VPR without visual distortion]{Negative log likelihood cost over mini-batch gradient descent iterations using VPR without visual distortion}
\label{fig:score-nolight-vpr}
\end{figure} %DONE
%%!TEX root = ../preamble.tex

\pgfplotstableread{data/light-vpr-shallow.dat}{\lightVPRShallow}
\pgfplotstableread{data/light-vpr-deep.dat}{\lightVPRDeep}

\begin{figure}[H]
\begin{tikzpicture}[scale=1]
	\begin{axis}[
			title=Cost over iterations,
			xlabel=Iterations,
			ylabel=Cost,
			xmin = -480,
			xmax = 12480,
			restrict x to domain=0:12000,
			xticklabel style={rotate=30},
			legend pos=north east,
		]
		\addplot+ [cred, mark=none] table [x={iteration}, y={score}] {\lightVPRShallow};
		\addlegendentry{Shallow network}
		\addplot+ [cblue, mark=none] table [x={iteration}, y={score}] {\lightVPRDeep};
		\addlegendentry{Deep network}
	\end{axis}
\end{tikzpicture}
\label{fig:score-light-vpr}
\caption[Training the VRC using VPR with visual distortion]{Negative log-likelihood cost over mini-batch gradient descent iterations using VPR with visual distortion}
\end{figure} %DONE

%%!TEX root = ../preamble.tex

\hspace{-8mm}
\begin{figure}[H]
\begin{tikzpicture}
\begin{axis}[
	width=0.67\textwidth,
	height=7cm,
	enlarge y limits={abs=0.5},
	xmin=0,
	xmax=100,
	axis y line*=none,
    axis x line*=bottom,
    xbar,
    reverse legend,
    legend style={at={(0.5,-0.2)},anchor=north},
    yticklabels={
		{Without visual distortion\\shallow topology},
		{Without visual distortion\\deep topology},
		{With visual distortion\\shallow topology},
		{With visual distortion\\deep topology},
	},
	yticklabel style={align=right},
    ytick=data,
    nodes near coords,
    every node near coord/.append style={
		/pgf/number format/fixed zerofill,
		/pgf/number format/precision=2
	},
    ]

% Test set
\addplot [draw=cgreen, fill=cgreen!70] coordinates {
	(97.39,3) % Correct value
	(78.95,2)
	(79.12,1)
	(79.84,0)
};

% Training set
\addplot [draw=cred, fill=cred!70] coordinates {
	(97.38,3) % Correct value
	(77.90,2)
	(79.61,1)
	(78.95,0)
};

\legend{Test set,Training set}
\end{axis}

\end{tikzpicture}
\caption{Accuracy of target detection using angular representation}
\label{fig:angular-acc}
\end{figure}

















































 %NEEDS CORRECT DATA
%%!TEX root = ../preamble.tex

%\hspace{-8mm}
\begin{figure}[H]
\begin{tikzpicture}
\begin{axis}[target-acc]

% Test set
\addplot [draw=cgreen, fill=cgreen!70] coordinates {
	(95.33,3) % Correct value - nolight shallow
	(95.19,2) % Correct Value - nolight deep
	(85.74,1) % Correct value - light shallow
	(86.66,0) % Correct value - light deep
};

% Training set
\addplot [draw=cred, fill=cred!70] coordinates {
	(96.05,3) % Correct value - nolight shallow
	(95.94,2) % Correct value - nolight deep
	(86.10,1) % Correct value - light shallow
	(86.60,0) % Correct value - light deep
};

\legend{Test set,Training set}
\end{axis}

\end{tikzpicture}
\caption{Accuracy of partitioning classification using VPR}
\label{fig:vpr-acc}
\end{figure}













































 %DONE

%%!TEX root = ../preamble.tex

\hspace{-8mm}
\begin{figure}[H]
\begin{tikzpicture}
\begin{axis}[
	width=0.67\textwidth,
	height=5cm,
	enlarge y limits={abs=0.5},
	xmin=0,
	xmax=0.065,
	axis y line*=none,
    axis x line*=bottom,
    xbar,
    reverse legend,
    legend style={at={(0.5,-0.1)},anchor=north},
    yticklabels={
		{Mean horizontal error},
		{Mean vertical error},
		{Mean distance error},
	},
	yticklabel style={align=right},
    ytick=data,
	xtick style={draw=none},	
    xticklabels={,,,,,},
    nodes near coords,
    every node near coord/.append style={
		/pgf/number format/fixed zerofill,
		/pgf/number format/precision=6
	},
    ]

% Test set
\addplot [draw=cgreen, fill=cgreen!70] coordinates {
	(0.030724,2) % horizontal error
	(0.036235,1) % vertical error
	(0.051281,0) % distance error
};

% Training set
\addplot [draw=cred, fill=cred!70] coordinates {
	(0.040724,2) % horizontal error
	(0.046235,1) % vertical error
	(0.061281,0) % distance error
};

% DEEP TOPOLOGY. model12.bin. 10,000 samples.
%Mean horizontal error is: 0.040724
%Mean vertical error is: 0.046235
%Mean distance error is: 0.061281
%Accuracy is: 0.979000

\legend{Test set,Training set}
\end{axis}

\end{tikzpicture}
\caption{Angular accuracy of the deep topology without visual distortion}
\label{fig:angular-acc}
\end{figure}
















































 %NEEDS CORRECT DATA
%%!TEX root = ../preamble.tex

\begin{figure}[h]
\begin{tikzpicture}
\begin{axis}[
	angular-mse,
]

% TEST SET
\addplot [draw=cgreen, fill=cgreen!70] coordinates {
	(0.096121,2) % Correct value
	(0.092301,1) % Correct value
	(0.164209,0) % Correct value
};

% TRAINING SET
\addplot [draw=cred, fill=cred!70] coordinates {
	(0.093103,2) % Correct value
	(0.087807,1) % Correct value
	(0.153207,0) % Correct value
};

\legend{Test set,Training set}
\end{axis}

\end{tikzpicture}
\caption[Mean error of the shallow CNN using AR without visual distortion]{Mean error of the shallow CNN without visual distortion}
\label{fig:angular-mse-nolight-shallow}
\end{figure}
















































 %DONE
%%!TEX root = ../preamble.tex

\hspace{-8mm}
\begin{figure}[H]
\begin{tikzpicture}
\begin{axis}[
	angular-mse,
]

\addplot [draw=cgreen, fill=cgreen!70] coordinates {
	(0.161777,2) % Correct value
	(0.172215,1) % Correct value
	(0.163934,0) % Correct value
};
\addplot [draw=cred, fill=cred!70] coordinates {
	(0.162300,2) % Correct value
	(0.171913,1) % Correct value
	(0.161433,0) % Correct value
};

\legend{Test set,Training set}
\end{axis}

\end{tikzpicture}
\caption{Angular accuracy of the deep topology with visual distortion}
\label{fig:angular-acc}
\end{figure}














































 %DONE
%%!TEX root = ../preamble.tex



\hspace{-8mm}
\begin{figure}[H]
\begin{tikzpicture}
\begin{axis}[
	angular-mse,
]

\addplot [draw=cgreen, fill=cgreen!70] coordinates {
	(0.179426,2) % Correct value
	(0.186698,1) % Correct value
	(0.169215,0) % Correct value
};
\addplot [draw=cred, fill=cred!70] coordinates {
	(0.179329,2) % Correct value
	(0.186092,1) % Correct value
	(0.165876,0) % Correct value
};

\legend{Test set,Training set}
\end{axis}

\end{tikzpicture}
\caption{Angular accuracy of the shallow topology with visual distortion}
\label{fig:angular-mse-light-shallow}
\end{figure}













































 %NEEDS CORRECT DATA



% NEAT GRAPHS
%%!TEX root = ../preamble.tex

\pgfplotstableread{data/neat-angular-fitness.dat}{\neatAngularFitness}
\pgfplotstableread{data/neat-vpr-fitness.dat}{\neatVPRFitness}

\begin{figure}[H]
\begin{tikzpicture}[scale=1]
	\begin{axis}[
			title=Overall fitness over generations,
			xlabel=Generations,
			ylabel=Overall fitness,
			xmin = -10,
			xmax = 225,
			restrict x to domain=0:215,
			xticklabel style={rotate=30},
			minor x tick num=1,
			legend pos=north west,
		]
		\addplot+ [cred, mark=none] table [x={Generation}, y={Overall}] {\neatAngularFitness};
		\addlegendentry{Angular representation}
		\addplot+ [cblue, mark=none] table [x={Generation}, y={Overall}] {\neatVPRFitness};
		\addlegendentry{Visual partitioning representation}
	\end{axis}
\end{tikzpicture}
\caption{Training the agent}
\end{figure} %DONE
%%!TEX root = ../preamble.tex

\pgfplotstableread{data/neat-angular-fitness.dat}{\neatAngularFitness}
\pgfplotstableread{data/neat-vpr-fitness.dat}{\neatVPRFitness}

\begin{figure}[H]
\begin{tikzpicture}[scale=1]
	\begin{axis}[
			title=Aiming fitness over generations,
			xlabel=Generations,
			ylabel=Aiming fitness,
			xmin = -10,
			xmax = 225,
			ymax = 85,
			restrict x to domain=0:215,
			xticklabel style={rotate=30},
			minor x tick num=1,
			legend pos=north west,
		]
		\addplot+ [cred, mark=none] table [x={Generation}, y={Aiming}] {\neatAngularFitness};
		\addlegendentry{Angular representation}
		\addplot+ [cblue, mark=none] table [x={Generation}, y={Aiming}] {\neatVPRFitness};
		\addlegendentry{Visual partitioning representation}
	\end{axis}
\end{tikzpicture}
\caption{Training the agent}
\end{figure} %DONE
%%!TEX root = ../preamble.tex

\pgfplotstableread{data/neat/ang-mean.dat}{\neatAngularMean}
\pgfplotstableread{data/neat-vpr-fitness.dat}{\neatVPRFitness}

\begin{figure}[H]
\begin{tikzpicture}[scale=1]
	\begin{axis}[
			title=Shooting fitness over generations,
			xlabel=Generations,
			ylabel=Shooting fitness,
			xmin = -10,
			xmax = 225,
			restrict x to domain=0:215,
			xticklabel style={rotate=30},
			minor x tick num=1,
			legend pos=north west,
		]
		\addplot+ [cred, mark=none] table [x={Generation}, y={ShootingFitness}] {\neatAngularMean};
		\addlegendentry{Angular representation}
		\addplot+ [cblue, mark=none] table [x={Generation}, y={Shooting}] {\neatVPRFitness};
		\addlegendentry{Visual partitioning representation}
	\end{axis}
\end{tikzpicture}
\caption{Training the agent}
\end{figure} %DONE

%%!TEX root = ../preamble.tex

\pgfplotstableread{data/neat/ang-mean.dat}{\neatAngularMean}

\begin{figure}[H]
\begin{tikzpicture}[scale=1]
	\begin{axis}[
			title=Fitness over generations,
			xlabel=Generations,
			ylabel=Fitness,
			xmin = -10,
			xmax = 225,
			restrict x to domain=0:215,
			xticklabel style={rotate=30},
			minor x tick num=1,
			legend pos=north west,
		]
		\addplot+ [bluered-mix, mark=none] table [x={Generation}, y={Fitness}] {\neatAngularMean};
		\addlegendentry{Overall fitness}
		\addplot+ [cred, mark=none] table [x={Generation}, y={AimingFitness}] {\neatAngularMean};
		\addlegendentry{Aiming fitness}
		\addplot+ [cblue, mark=none] table [x={Generation}, y={ShootingFitness}] {\neatAngularMean};
		\addlegendentry{Shooting fitness}
	\end{axis}
\end{tikzpicture}
\caption{Training the angular agent}
\end{figure} %DONE
%%!TEX root = ../preamble.tex

\pgfplotstableread{data/neat-vpr-fitness.dat}{\neatVPRFitness}

\begin{figure}[H]
\begin{tikzpicture}[scale=1]
	\begin{axis}[
			title=Fitness over generations,
			xlabel=Generations,
			ylabel=Fitness,
			xmin = -10,
			xmax = 225,
			restrict x to domain=0:215,
			xticklabel style={rotate=30},
			minor x tick num=1,
			legend pos=north west,
		]
		\addplot+ [bluered-mix, mark=none] table [x={Generation}, y={Overall}] {\neatVPRFitness};
		\addlegendentry{Overall fitness}
		\addplot+ [cred, mark=none] table [x={Generation}, y={Aiming}] {\neatVPRFitness};
		\addlegendentry{Aiming fitness}
		\addplot+ [cblue, mark=none] table [x={Generation}, y={Shooting}] {\neatVPRFitness};
		\addlegendentry{Shooting fitness}
	\end{axis}
\end{tikzpicture}
\caption{Training the VPR agent}
\end{figure} %DONE





























































