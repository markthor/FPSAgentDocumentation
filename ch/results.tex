%!TEX root = ../preamble.tex

\section{Results}
This section describes the results of the experiments explained in section \ref{sec:experiments}.
%!TEX root = ../preamble.tex

\pgfplotsset{max space between ticks=40}

% USED FOR TARGET DETECTION ACCURACY
\pgfplotsset{target-acc/.style={%
        width=0.67\textwidth,
		height=7cm,
		enlarge y limits={abs=.625},
		xmin=0,
		xmax=100,
		axis y line*=none,
	    axis x line*=bottom,
	    xbar,
	    reverse legend,
	    legend columns=2,
		legend style={
			at={(0.5,-0.125)},
			anchor=north,
			/tikz/column 2/.style={
				column sep=10pt,
			}
		},
	    yticklabels={
	    		{Without visual distortion,\\shallow network},
			{Without visual distortion,\\deep network},
			{With visual distortion,\\shallow network},
			{With visual distortion,\\deep network},
		},
		yticklabel style={align=right},
		ytick=data,
	    nodes near coords,
		every node near coord/.append style={
			/pgf/number format/fixed zerofill,
			/pgf/number format/precision=2
		},
	}
}

% USED FOR ANGULAR MSE
\pgfplotsset{angular-mse/.style={%
		width=0.67\textwidth,
		height=5.5cm,
		enlarge y limits={abs=0.7},
		xmin=0,
		xmax=0.2,
		axis y line*=none,
		axis x line*=bottom,
	    xbar,
	    reverse legend,
		%legend style={at={(0.5,-0.2)},anchor=north},
		legend columns=2,
		legend style={
			at={(0.5,-0.2)},
			anchor=north,
			/tikz/column 2/.style={
				column sep=10pt,
			}
		},
	    yticklabels={
			{Mean horizontal error},
			{Mean vertical error},
			{Mean distance error},
		},
		yticklabel style={align=right},
	    ytick=data,
		%xtick style={draw=none},	
	    %xticklabels={,,,,,},
	    nodes near coords,
	    every node near coord/.append style={
			/pgf/number format/fixed zerofill,
			/pgf/number format/precision=6
		},
	}
}




































\label{sec:results}

\subsection{Convolutional neural network experiments}
\subsubsection{Visual partitioning representation}
The performance of the VPR CNN is measured as the cost function on the mini-batch that the gradient is estimated from in mini-batch gradient descent. As the process of batch selection is random, the score is fluctuating. The cost function is negative log-likelihood, described in section \ref{sec:negative}.
The results presented are with and without visual distortion and with the network topologies described in section \ref{sec:topologies}.

%!TEX root = ../preamble.tex

\pgfplotstableread{data/nolight-vpr-shallow.dat}{\nolightVPRShallow}
\pgfplotstableread{data/nolight-vpr-deep.dat}{\nolightVPRDeep}

\begin{figure}[H]
\begin{tikzpicture}[scale=1]
	\begin{axis}[
			title=Cost over iterations,
			xlabel=Iterations,
			ylabel=Cost,
			xmin = -480,
			xmax = 12480,
			restrict x to domain=0:12000,
			xticklabel style={rotate=30},
			legend pos=north east,
		]
		\addplot+ [cred, mark=none] table [x={iteration}, y={score}] {\nolightVPRShallow};
		\addlegendentry{Shallow network}
		\addplot+ [cblue, mark=none] table [x={iteration}, y={score}] {\nolightVPRDeep};
		\addlegendentry{Deep network}
	\end{axis}
\end{tikzpicture}
\caption[Training the VRC using VPR without visual distortion]{Negative log-likelihood cost over mini-batch gradient descent iterations using VPR without visual distortion}
\end{figure} %DONE
%!TEX root = ../preamble.tex

\pgfplotstableread{data/light-vpr-shallow.dat}{\lightVPRShallow}
\pgfplotstableread{data/light-vpr-deep.dat}{\lightVPRDeep}

\begin{figure}[H]
\begin{tikzpicture}[scale=1]
	\begin{axis}[
			title=Cost over iterations,
			xlabel=Iterations,
			ylabel=Cost,
			xmin = -480,
			xmax = 12480,
			restrict x to domain=0:12000,
			xticklabel style={rotate=30},
			legend pos=north east,
		]
		\addplot+ [cred, mark=none] table [x={iteration}, y={score}] {\lightVPRShallow};
		\addlegendentry{Shallow network}
		\addplot+ [cblue, mark=none] table [x={iteration}, y={score}] {\lightVPRDeep};
		\addlegendentry{Deep network}
	\end{axis}
\end{tikzpicture}
\caption[Training the VRC using VPR with visual distortion]{Negative log-likelihood cost over mini-batch gradient descent iterations using VPR with visual distortion}
\end{figure} %DONE

\noindent
The graphs show that the shallow topology converges in fewer iterations, but both networks manage to converge to a solution. 

%!TEX root = ../preamble.tex

\hspace{-8mm}
\begin{figure}[H]
\begin{tikzpicture}
\begin{axis}[
	%title=Accuracy of different setups,
	width=0.67\textwidth,
	height=8cm,
	enlarge y limits={abs=.625},
	xmin=0,
	xmax=100,
	axis y line*=none,
    axis x line*=bottom,
    xbar,
    reverse legend,
    legend style={at={(0.5,-0.1)},anchor=north},
    yticklabels={
		{With visual distortion,\\shallow architecture},
		{With visual distortion,\\deep architecture},
		{Without visual distortion,\\shallow architecture},
		{Without visual distortion,\\deep architecture},
	},
	yticklabel style={align=right},
    ytick=data,
    nodes near coords,
    every node near coord/.append style={
		/pgf/number format/fixed zerofill,
		/pgf/number format/precision=2
	},
    ]

% Test set
\addplot [draw=cgreen, fill=cgreen!70] coordinates {
	(85.74,3) % Correct value
	(86.66,2) % Correct value
	(95.33,1) % Correct value
	(95.19,0) % Correct Value
};


% Training set
\addplot [draw=cred, fill=cred!70] coordinates {
	(86.10,3) % Correct value
	(86.6,2) % Correct value
	(96.05,1) % Correct value
	(95.94,0) % Correct value
};

\legend{Test set,Training set}
\end{axis}

\end{tikzpicture}
\caption{Accuracy of target detection using visual partitioning}
\label{fig:vpr-acc}
\end{figure}
















































 %DONE

\noindent
The accuracy is measured as the percentage of correct predictions. It is apparent from these results, that the models have not overfitted to the training data, as the difference in accuracy of the training set and the test set is insignificant. Furthermore, the topologies of the networks does not seem to have a significant impact on the accuracy. Examples of the training examples that the deep CNN with visual distortion fails to classify correctly can be seen in section \ref{sec:incorrectpredictions} of the appendix. The incorrect predictions are due to the target being in between partitions or behind the weapon overlay.

%%!TEX root = ../preamble.tex

\pgfplotstableread{data/angular-nolight-shallow.dat}{\angularNolightShallow}
\pgfplotstableread{data/angular-nolight-deep.dat}{\angularNolightDeep}

\begin{figure}[H]
\begin{tikzpicture}[scale=1]
	\begin{axis}[
			title=Cost over iterations,
			xlabel=Iterations,
			ylabel=Cost,
			ymin = 0,
			ymax = 0.6,
			xmin = -640,
			xmax = 16640,
			restrict x to domain=0:16000,
			xticklabel style={rotate=30},
			minor x tick num=1,
			legend pos=north east,
		]
		\addplot+ [cred, mark=none] table [x={iteration}, y={score}] {\angularNolightShallow};
		\addlegendentry{Shallow network}
		\addplot+ [cblue, mark=none] table [x={iteration}, y={score}] {\angularNolightDeep};
		\addlegendentry{Deep network}
	\end{axis}
\end{tikzpicture}
\caption{Mean squared error cost over mini-batch gradient descent iterations using AR without visual distortion}
\label{fig:score-nolight-angular}
\end{figure}
%%!TEX root = ../preamble.tex

\pgfplotstableread{data/angular-light-shallow.dat}{\angularLightShallow}
\pgfplotstableread{data/angular-light-deep.dat}{\angularLightDeep}

\begin{figure}[H]
\begin{tikzpicture}[scale=1]
	\begin{axis}[
			height=7.5cm,
			%title=Cost over iterations,
			xlabel=Iterations,
			ylabel=Cost,
			ymin=0,
			ymax=0.55,
			xmin = -640,
			xmax = 16640,
			restrict x to domain=0:16000,
			xticklabel style={rotate=30},
			minor x tick num=1,
			legend pos=north east,
		]
		\addplot+ [cred, mark=none] table [x={iteration}, y={score}] {\angularLightShallow};
		\addlegendentry{Shallow network}
		\addplot+ [cblue, mark=none] table [x={iteration}, y={score}] {\angularLightDeep};
		\addlegendentry{Deep network}
	\end{axis}
\end{tikzpicture}
\caption{Mean squared error cost over mini-batch gradient descent iterations using AR with visual distortion}
\label{fig:score-light-angular}
\end{figure}

%%!TEX root = ../preamble.tex

\begin{figure}[h]
\vspace{15mm}
\begin{tikzpicture}
\begin{axis}[target-acc]

% Test set
\addplot [draw=cgreen, fill=cgreen!70] coordinates {
	(97.39,3) % Correct value - nolight shallow
	(97.39,2) % Correct value - nolight deep
	(89.60,1) % Correct value - light shallow
	(89.78,0) % Correct value - light deep
};

% Training set
\addplot [draw=cred, fill=cred!70] coordinates {
	(97.39,3) % Correct value - nolight shallow
	(97.38,2) % Correct value - nolight deep
	(90.20,1) % Correct value - light shallow
	(90.56,0) % Correct value - light deep
};

\legend{Test set,Training set}
\end{axis}

\end{tikzpicture}
\caption{Accuracy of target detection using AR}
\label{fig:angular-acc}
\end{figure}

















































 %DONE

%%!TEX root = ../preamble.tex

\hspace{-8mm}
\begin{figure}[H]
\begin{tikzpicture}
\begin{axis}[
	width=0.67\textwidth,
	height=5cm,
	enlarge y limits={abs=0.5},
	xmin=0,
	xmax=0.065,
	axis y line*=none,
    axis x line*=bottom,
    xbar,
    reverse legend,
    legend style={at={(0.5,-0.1)},anchor=north},
    yticklabels={
		{Mean horizontal error},
		{Mean vertical error},
		{Mean distance error},
	},
	yticklabel style={align=right},
    ytick=data,
	xtick style={draw=none},	
    xticklabels={,,,,,},
    nodes near coords,
    every node near coord/.append style={
		/pgf/number format/fixed zerofill,
		/pgf/number format/precision=6
	},
    ]

% Test set
\addplot [draw=cgreen, fill=cgreen!70] coordinates {
	(0.030724,2) % horizontal error
	(0.036235,1) % vertical error
	(0.051281,0) % distance error
};

% Training set
\addplot [draw=cred, fill=cred!70] coordinates {
	(0.040724,2) % horizontal error
	(0.046235,1) % vertical error
	(0.061281,0) % distance error
};

% DEEP TOPOLOGY. model12.bin. 10,000 samples.
%Mean horizontal error is: 0.040724
%Mean vertical error is: 0.046235
%Mean distance error is: 0.061281
%Accuracy is: 0.979000

\legend{Test set,Training set}
\end{axis}

\end{tikzpicture}
\caption{Angular accuracy of the deep topology without visual distortion}
\label{fig:angular-acc}
\end{figure}
















































 %DONE
%%!TEX root = ../preamble.tex

\hspace{-8mm}
\begin{figure}[H]
\begin{tikzpicture}
\begin{axis}[
	width=0.67\textwidth,
	height=5cm,
	enlarge y limits={abs=0.5},
	xmin=0,
	xmax=0.1,
	axis y line*=none,
    axis x line*=bottom,
    xbar,
    reverse legend,
    legend style={at={(0.5,-0.1)},anchor=north},
    yticklabels={
		{Mean horizontal error},
		{Mean vertical error},
		{Mean distance error},
	},
	yticklabel style={align=right},
    ytick=data,
	xtick style={draw=none},	
    xticklabels={,,,,,},
    nodes near coords,
    every node near coord/.append style={
		/pgf/number format/fixed zerofill,
		/pgf/number format/precision=6
	},
    ]

% TEST SET
\addplot [draw=cgreen, fill=cgreen!70] coordinates {
	(0.055178,2)
	(0.057934,1)
	(0.095012,0)
};

% TRAINING SET
\addplot [draw=cred, fill=cred!70] coordinates {
	(0.052970,2)
	(0.055122,1)
	(0.089300,0)
};

\legend{Test set,Training set}
\end{axis}

\end{tikzpicture}
\caption{Angular accuracy of the shallow topology without visual distortion}
\label{fig:angular-acc}
\end{figure}
















































 %DONE
%%!TEX root = ../preamble.tex

\hspace{-8mm}
\begin{figure}[H]
\begin{tikzpicture}
\begin{axis}[
	width=0.67\textwidth,
	height=5cm,
	enlarge y limits={abs=0.5},
	xmin=0,
	xmax=0.175,
	axis y line*=none,
    axis x line*=bottom,
    xbar,
    reverse legend,
    legend style={at={(0.5,-0.1)},anchor=north},
    yticklabels={
		{Mean horizontal error},
		{Mean vertical error},
		{Mean distance error},
	},
	yticklabel style={align=right},
    ytick=data,
	xtick style={draw=none},	
    xticklabels={,,,,,},
    nodes near coords,
    every node near coord/.append style={
		/pgf/number format/fixed zerofill,
		/pgf/number format/precision=6
	},
    ]

\addplot [draw=cgreen, fill=cgreen!70] coordinates {
	(0.001243,2)
	(0.002089,1)
	(0.001004,0)
};
\addplot [draw=cred, fill=cred!70] coordinates {
	(0.162300,2) % Correct value
	(0.171913,1) % Correct value
	(0.161433,0) % Correct value
};

\legend{Test set,Training set}
\end{axis}

\end{tikzpicture}
\caption{Angular accuracy of the deep topology with visual distortion}
\label{fig:angular-acc}
\end{figure}
















































 %DONE
%%!TEX root = ../preamble.tex

\hspace{-8mm}
\begin{figure}[H]
\begin{tikzpicture}
\begin{axis}[
	angular-mse,
	xmax=0.19,
]

\addplot [draw=cgreen, fill=cgreen!70] coordinates {
	(0.001243,2)
	(0.002089,1)
	(0.001004,0)
};
\addplot [draw=cred, fill=cred!70] coordinates {
	(0.179329,2) % Correct value
	(0.186092,1) % Correct value
	(0.165876,0) % Correct value
};

\legend{Test set,Training set}
\end{axis}

\end{tikzpicture}
\caption{Angular accuracy of the shallow topology with visual distortion}
\label{fig:angular-acc}
\end{figure}















































 %DONE

%%!TEX root = ../preamble.tex

%\pgfplotstableread{data/angular-light-shallow.dat}{\angularLightShallow}
\pgfplotstableread{data/angular-light-deep-small-dataset-50.dat}{\angularLightDeepSmallDatasetFifty}

\begin{figure}[H]
\begin{tikzpicture}[scale=1]
	\begin{axis}[
			title=Score over iterations,
			xlabel=Iterations,
			ylabel=Score,
			ymin=0,
			ymax=0.55,
			xmin = -166,
			xmax = 4306,
			restrict x to domain=0:4140,
			xticklabel style={rotate=30},
			minor x tick num=1,
			legend pos=north east,
		]
%		\addplot+ [cred, mark=none] table [x={iteration}, y={score}] {\angularLightShallow};
%		\addlegendentry{Shallow topology}
		\addplot+ [cblue, mark=none] table [x={iteration}, y={score}] {\angularLightDeepSmallDatasetFifty};
		\addlegendentry{Deep topology}
	\end{axis}
\end{tikzpicture}
\caption{Training convolutional neural networks using angular regression with visual distortion}
\end{figure}
%%!TEX root = ../preamble.tex

\hspace{-8mm}
\begin{figure}[H]
\begin{tikzpicture}
\begin{axis}[
	angular-mse,
	xmax=0.45,
]

\addplot [draw=cgreen, fill=cgreen!70] coordinates {
	(0.434460,2) % Correct value
	(0.379895,1) % Correct value
	(0.202830,0) % Correct value
};
\addplot [draw=cred, fill=cred!70] coordinates {
	(0.229095,2) % Correct value
	(0.217716,1) % Correct value
	(0.090129,0) % Correct value
};

\legend{Test set,Training set}
\end{axis}

\end{tikzpicture}
\caption{Angular accuracy of the deep topology with visual distortion using training data that has 50 samples per class}
\label{fig:angular-acc}
\end{figure}
































\subsubsection{Feature maps}
The feature maps shown on figure \ref{fig:featuremaps} are visualised by scaling the output range of $[0,1]$ of every neuron in the convolutional layers linearly to the grey scale range of $[0,255]$. The leftmost column of feature maps are from the first convolutional layer, and the rightmost is the input to the fully connected layers. As a convolutional layer takes a depth slice of all the previous feature maps as input, their is no apparent connection between the visualised output of the max pooling layer and the following result of the convolutional layer.

\begin{figure}[H]
	\begin{scriptsize}
		\sffamily
		\def\svgwidth{\textwidth}
		\input{img/featuremaps.pdf_tex}
	\end{scriptsize}
	\caption[Feature maps]{A small subset of the feature maps produced from running a training example from the lightened arena through the visual partitioning classification deep convolutional neural network. The feature maps highlight the position of the target.}
	\label{fig:featuremaps}
\end{figure}

% NEAT GRAPHS
%!TEX root = ../preamble.tex

\pgfplotstableread{data/neat/ang-mean.dat}{\neatAngularMean}
\pgfplotstableread{data/neat-vpr-fitness.dat}{\neatVPRFitness}

\begin{figure}[H]
\begin{tikzpicture}[scale=1]
	\begin{axis}[
			title=Overall fitness over generations,
			xlabel=Generations,
			ylabel=Overall fitness,
			xmin = -10,
			xmax = 225,
			restrict x to domain=0:215,
			xticklabel style={rotate=30},
			minor x tick num=1,
			legend pos=north west,
		]
		\addplot+ [cred, mark=none] table [x={Generation}, y={Fitness }] {\neatAngularMean};
		\addlegendentry{Angular representation}
		\addplot+ [cblue, mark=none] table [x={Generation}, y={Overall}] {\neatVPRFitness};
		\addlegendentry{Visual partitioning representation}
	\end{axis}
\end{tikzpicture}
\caption{Training the agent}
\end{figure} %DONE
%!TEX root = ../preamble.tex

\pgfplotstableread{data/neat/ang-mean.dat}{\neatAngularMean}
\pgfplotstableread{data/neat-vpr-fitness.dat}{\neatVPRFitness}

\begin{figure}[H]
\begin{tikzpicture}[scale=1]
	\begin{axis}[
			title=Aiming fitness over generations,
			xlabel=Generations,
			ylabel=Aiming fitness,
			xmin = -10,
			xmax = 225,
			ymax = 85,
			restrict x to domain=0:215,
			xticklabel style={rotate=30},
			minor x tick num=1,
			legend pos=north west,
		]
		\addplot+ [cred, mark=none] table [x={Generation}, y={AimingFitness}] {\neatAngularMean};
		\addlegendentry{Angular representation}
		\addplot+ [cblue, mark=none] table [x={Generation}, y={Aiming}] {\neatVPRFitness};
		\addlegendentry{Visual partitioning representation}
	\end{axis}
\end{tikzpicture}
\caption{Training the agent}
\end{figure} %DONE
%!TEX root = ../preamble.tex

\pgfplotstableread{data/neat-angular-fitness.dat}{\neatAngularFitness}
\pgfplotstableread{data/neat-vpr-fitness.dat}{\neatVPRFitness}

\begin{figure}[H]
\begin{tikzpicture}[scale=1]
	\begin{axis}[
			title=Shooting fitness over generations,
			xlabel=Generations,
			ylabel=Shooting fitness,
			xmin = -10,
			xmax = 225,
			restrict x to domain=0:215,
			xticklabel style={rotate=30},
			minor x tick num=1,
			legend pos=north west,
		]
		\addplot+ [cred, mark=none] table [x={Generation}, y={Shooting}] {\neatAngularFitness};
		\addlegendentry{Angular representation}
		\addplot+ [cblue, mark=none] table [x={Generation}, y={Shooting}] {\neatVPRFitness};
		\addlegendentry{Visual partitioning representation}
	\end{axis}
\end{tikzpicture}
\caption{Training the agent}
\end{figure} %DONE

%!TEX root = ../preamble.tex

\pgfplotstableread{data/neat-angular-fitness.dat}{\neatAngularFitness}

\begin{figure}[H]
\begin{tikzpicture}[scale=1]
	\begin{axis}[
			title=Fitness over generations,
			xlabel=Generations,
			ylabel=Fitness,
			xmin = -10,
			xmax = 225,
			restrict x to domain=0:215,
			xticklabel style={rotate=30},
			minor x tick num=1,
			legend pos=north west,
		]
		\addplot+ [bluered-mix, mark=none] table [x={Generation}, y={Overall}] {\neatAngularFitness};
		\addlegendentry{Overall fitness}
		\addplot+ [cred, mark=none] table [x={Generation}, y={Aiming}] {\neatAngularFitness};
		\addlegendentry{Aiming fitness}
		\addplot+ [cblue, mark=none] table [x={Generation}, y={Shooting}] {\neatAngularFitness};
		\addlegendentry{Shooting fitness}
	\end{axis}
\end{tikzpicture}
\caption{Training the angular agent}
\end{figure} %DONE
%!TEX root = ../../preamble.tex

\pgfplotstableread{data/neat/vpr-mean.dat}{\neatVPRMean}

\begin{figure}[H]
\begin{tikzpicture}[scale=1]
	\begin{axis}[
			title=Fitness over generations,
			xlabel=Generations,
			ylabel=Fitness,
			ymin = 0,
			xmin = -10,
			xmax = 225,
			restrict x to domain=0:215,
			xticklabel style={rotate=30},
			minor x tick num=1,
			legend pos=north west,
		]
		\addplot+ [bluered-mix, mark=none] table [x={Generation}, y={Fitness}] {\neatVPRMean};
		\addlegendentry{Overall fitness}
		\addplot+ [cred, mark=none] table [x={Generation}, y={AimingFitness}] {\neatVPRMean};
		\addlegendentry{Aiming fitness}
		\addplot+ [cblue, mark=none] table [x={Generation}, y={ShootingFitness}] {\neatVPRMean};
		\addlegendentry{Shooting fitness}
	\end{axis}
\end{tikzpicture}
\caption{Training the VPR agent without recoil}
\end{figure} %DONE





























































