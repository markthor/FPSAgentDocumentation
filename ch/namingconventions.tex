%!TEX root = ../preamble.tex

\section{Naming conventions}

\begin{table}[H]
\begin{center}

\begin{tabularx}{\textwidth}{ | p{5cm} | X |}
		\hline
	
		\textbf{Name} & \textbf{Definition} \\ \hline
		Visual recognition component (VRC) & The component responsible for creating a feature representation of an image. It is implemented as a convolutional neural network in this project. \\ \hline
		Action inferring component (AIC) & The component responsible for inferring actions from the feature representation of the visual recognition component. This is implemented as a neural network evolved with neuroevolution in this project. \\ \hline
		
		Pipeline & The combination of the VRC and the AIC component, translating the visual state to action. \\ \hline
		
Translates per second (TPS) & The number of times per second the state of the game is passed through the AIC or the pipeline to infer an action.\\ \hline		
		
		Visual partitioning representation (VPR) & A representation of the visual state, based on partition classes, described in section \ref{sec:vpr}. \\ \hline
		
		Angular representation (AR) & A representation of the visual state based on relative angles between the agent and the target, described in section \ref{sec:angular}. \\ \hline
		Agent & The game entity that the developed AI controls. \\ \hline
		Target & The game entity that the agent is rewarded for shooting. \\ \hline
		
		Tap-fire & Shooting more accurate with a short delay between shots in order to reduce the penalty that weapon recoil has on accuracy. \\ \hline
		
		
\end{tabularx}
\end{center}
\label{tab:naming-conventions} 
\end{table}

\newpage