%!TEX root = ../preamble.tex

\newpage
\section{Future work}
\label{sec:futurework}
The integration of the VRC and AIC involved binarising the output of the VRC, such that it mimics the output that the AIC was trained with. However, this does not explore the potential of the AIC to use the analog output of the CNN, which might make the integration work better. Neuroevolution can possibly learn how to respond to the uncertainties and incorrect classifications of the VRC, and perform better than the combined agent developed in this project. The probability distribution of the features vector of the VPR reveals additional information about the position of the target, as an uncertainty usually means that the target is between two partitions, as exemplified in figure \ref{fig:uncertain}. Consequently, we believe that evolving the ANN with the analog output of the CNN could result in even better performance than using the ground truths directly.

\begin{figure}[H]
	\centering
	\begin{scriptsize}
		\sffamily
		\input{img/uncertain.pdf_tex}
	\end{scriptsize}
	\caption[An example of uncertain classification]{An example of a relatively low prediction confidence of the CNN, indicating that the target is in between partitions.}
	\label{fig:uncertain}
\end{figure}
\noindent
Using HyperNEAT as by Kenneth O. Stanley et al.\cite{DBLP:journals/alife/StanleyDG09}, to evolve the AIC using the VPR, or even directly on feature maps, as seen in section~\ref{sec:featuremaps}, could potentially increase the evolution speed and the resulting performance. HyperNEAT effectively evolves large scale neural network on spatially related inputs, and the dimensions of both the feature maps and the VPR are spatially related.