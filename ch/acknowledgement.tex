%!TEX root = ../preamble.tex
\newgeometry{left=4.5cm, right=4.5cm}

\begin{center}
\section*{Foreword}
\vspace{1em}
\end{center}


\noindent
We would like to thank our supervisor Sebastian Risi\footnote{\url{http://sebastianrisi.com/}} for guidance throughout the project.

A special thanks has to be given to Muhsin Kaymak for his priceless help with getting the Unity framework to suit our needs.

The Deep Learning for Java\footnote{\url{https://deeplearning4j.org/}} community has given much appreciated assistance with the DL4J framework - especially raver119\footnote{\url{https://github.com/raver119}}, AlexDBlack\footnote{\url{https://github.com/AlexDBlack}} and agibsonccc\footnote{\url{https://github.com/agibsonccc}}.

\vspace{5mm}

\noindent
The code for the project can be found at \url{https://github.com/Prechtig/FPSAgent} and is published under the GNU General Public License v. 3.0\footnote{\url{https://www.gnu.org/licenses/gpl-3.0.en.html}}.

\vspace{5mm}

\noindent
Four demonstration videos are available, showcasing:
\begin{itemize}
  \item AR with ground truths at \url{youtu.be/-7dXk2JVJ_4}
  \item AR with VRC at \url{youtu.be/3qO9vd3SZqM}
  \item VPR with ground truths at \url{youtu.be/6PqhAdITsZo}
  \item VPR with VRC and a heatmap overlay, depicting the VRC's classification at \url{youtu.be/UNZIhHow4iw} 
\end{itemize}
Section~\ref{sec:naming-conventions} explains the abbreviations AR, VPR and VRC.

\restoregeometry








