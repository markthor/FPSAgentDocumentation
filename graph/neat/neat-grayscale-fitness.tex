%!TEX root = ../../preamble.tex

\pgfplotstableread{data/grayscale/grayscale.dat}{\neatGrayscaleScore}

\begin{figure}[H]
\begin{tikzpicture}[scale=1]
	\begin{axis}[
			height=8cm,
			width=0.95\textwidth,
			title=Fitness per generation,
			xlabel=Generations,
			ylabel=Fitness,
			ymin = -5,
			ymax = 75,
			xmin = -4,
			xmax = 104,
			restrict x to domain=0:100,
			xticklabel style={rotate=30},
			minor tick num=1,
			legend pos=north west,
			reverse legend,
			%transpose legend,
			%legend columns=2,
			%legend style={
			%	/tikz/column 2/.style={
			%		column sep=10pt,
			%	}
			%},
		]
		\addplot [cblue, mark=none] table [x={Generation}, y={ShootingFitness}] {\neatGrayscaleScore};
		\addlegendentry{Shooting fitness}
		\addplot [cred, mark=none] table [x={Generation}, y={AimingFitness}] {\neatGrayscaleScore};
		\addlegendentry{Aiming fitness}
		\addplot [cpurple, mark=none] table [x={Generation}, y={Fitness}] {\neatGrayscaleScore};
		\addlegendentry{Overall fitness}
	\end{axis}
\end{tikzpicture}
\caption[Fitness from direct visual input 28x28 greyscaled]{Fitness from training an agent with neuroevolution using direct visual input scaled to 28x28 and greyscaled using visual distortion without recoil. The approach is the same as the neuroevolution experiments in section~\ref{sec:experiments}, except that the input is raw pixels.}
\label{fig:neat-grayscale}
\end{figure}
