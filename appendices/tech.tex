%!TEX root = ../preamble.tex

\section{Technical Description}
Bla bla bla

\subsection{Hardware used for training}
\label{sub:hardware}
All training of the CNN was done on a server, at the IT University of Copenhagen, with the following specs:

\begin{description}
	\item [CPU] Intel\textsuperscript{\textregistered} Core\textsuperscript{\texttrademark} i7-5820K
	\item [GPU] 2x GeForce\textsuperscript{\textregistered} GTX TITAN X
	\item [RAM] 32 GB HyperX Fury DDR4 2400 C15
\end{description}	

\noindent
Each training experiment was done using one of the GPUs and up to 12 GB of RAM allowing for two simultaneous runs.

\subsection{Bridging between Unity and DL4J}\todo{Move this to Theory?}
The game used is made using the game engine Unity, where it is possible to interface with the game using either C\#, JavaScript, Boo or Unity Script.

The convolutional neural network is trained using DL4J \todo{Has DL4J been introduced at this point?}, which is written in Java.
It is unfortunately not possible to interface with the game using Java. Somehow the game has to communicate with a convolutional neural network.

\section{Figures}

\begin{figure}[H]
	\begin{scriptsize}
		\sffamily
		\def\svgwidth{\textwidth}
		\input{img/grid.pdf_tex}
	\end{scriptsize}
	\caption{Ids of the 3, 3, 3 partitioning scheme}
	\label{fig:ids}
\end{figure}

\section{Incorrect predictions}
\label{sec:incorrectpredictions}
The following examples are incorrect predictions by the deep visual partitioning convolutional neural network. The green squares mark the ground truths, and the red squares mark the wrong predictions. Absence of either the red or the green square means that either the the prediction  or the ground truth is that no target is present on the screen.

\begin{figure}[H]
	\begin{center}
	\begin{scriptsize}
		\sffamily
		\def\svgwidth{0.95\textwidth}
		\input{img/failcollection1.pdf_tex}
	\end{scriptsize}
	\label{fig:failcollection1}
	\end{center}
\end{figure}


\begin{figure}[H]
	\begin{center}
	\begin{scriptsize}
		\sffamily
		\def\svgwidth{0.95\textwidth}
		\input{img/failcollection2.pdf_tex}
	\end{scriptsize}
	\label{fig:failcollection2}
	\end{center}
\end{figure}

\section{Angular representation error}
\label{sec:angular-error}
\begin{figure}[H]
	\begin{center}
	\begin{scriptsize}
		\sffamily
		\def\svgwidth{1\textwidth}
		\input{img/angularError.pdf_tex}
	\end{scriptsize}
	\label{fig:angularerror}
	\caption[AR mean error visualised]{The blue box marks the average error margins for horizontal and vertical angles for the deep CNN without visual distortion, while the red box marks the same error for the shallow CNN.}
	\end{center}
\end{figure}
