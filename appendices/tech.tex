%!TEX root = ../preamble.tex

\section{Technical Description}
\subsection{Hardware used for training}
\label{sub:hardware}
All training of the CNN was done on a server, at the IT University of Copenhagen, with the following specs:

\begin{description}
	\item [CPU] Intel\textsuperscript{\textregistered} Core\textsuperscript{\texttrademark} i7-5820K
	\item [GPU] 2x GeForce\textsuperscript{\textregistered} GTX TITAN X
	\item [RAM] 32 GB HyperX Fury DDR4 2400 C15
\end{description}	

\noindent
Each training experiment was done using one of the GPUs and up to 12 GB of RAM allowing for two simultaneous runs.

\subsection{Communication between the action inferring component and the visual recognition component}
\label{sec:com-vrc-aic}
The game used throughout the project was made using the game engine Unity, where it is possible to interface with the game using either C\#, JavaScript, Boo or Unity Script. Since the game has to send the pixel data to a CNN to obtain the inputs for the AIC, the easiest option would be to implement the VRC in one of the languages supported by Unity.

Unfortunately we were unable to find a framework suitable for our needs, why we decided to look for frameworks in other languages. Java is the language we are most comfortable in, why we decided to use a Java framework, called DeepLearning4J\footnote{\url{https://deeplearning4j.org/}}.

Neither of the languages used to interface with the game can natively communicate with Java. In order to allow communication between the VRC and the AIC a simple socket bridge was implemented. The overhead introduced by this bridge is minimal as it is possible to send the required data back and forth 10,000 times in approximately three seconds.

\section{Incorrect predictions}
\label{sec:incorrectpredictions}
The following examples are incorrect predictions by the deep visual partitioning convolutional neural network. The green squares mark the ground truths, and the red squares mark the wrong predictions. Absence of either the red or the green square means that either the the prediction  or the ground truth is that no target is present on the screen.

\begin{figure}[H]
	\begin{center}
	\begin{scriptsize}
		\sffamily
		\def\svgwidth{0.95\textwidth}
		\input{img/failcollection1.pdf_tex}
	\end{scriptsize}
	\end{center}
\end{figure}


\begin{figure}[H]
	\begin{center}
	\begin{scriptsize}
		\sffamily
		\def\svgwidth{0.95\textwidth}
		\input{img/failcollection2.pdf_tex}
	\end{scriptsize}
	\end{center}
\end{figure}

\section{Angular representation error}
\label{sec:angular-error}
\begin{figure}[H]
	\begin{center}
	\begin{scriptsize}
		\sffamily
		\def\svgwidth{1\textwidth}
		\input{img/angularError.pdf_tex}
	\end{scriptsize}
	\end{center}
	\caption[AR mean error visualised]{The blue box marks the average error margins for horizontal and vertical angles for the deep CNN without visual distortion, while the red box marks the same error for the shallow CNN.}
	\label{fig:angularerror}
\end{figure}

\begin{figure}[H]
	\begin{center}
	\begin{scriptsize}
		\sffamily
		\def\svgwidth{0.9\textwidth}
		\input{img/aecollection1.pdf_tex}
	\end{scriptsize}
	\end{center}
	\caption[AR error examples visualised]{The green dot marks the correct position and the red dot marks the estimated position using the deep network.}
	\label{fig:aecollection}
\end{figure}

\section{Visually distorted examples}
\label{sec:vdexamples}
\begin{figure}[H]
	\begin{center}
	\begin{scriptsize}
		\sffamily
		\input{img/vd1.pdf_tex}
	\end{scriptsize}
	\end{center}
	\caption[Visually distorted example]{The deep CNN estimating the VPR correctly predicts the class(id 0) with a confidence of 71.9\%.}
	\label{fig:vd1}
\end{figure}

\begin{figure}[H]
	\begin{center}
	\begin{scriptsize}
		\sffamily
		\input{img/vd2.pdf_tex}
	\end{scriptsize}
	\end{center}
	\caption[Visually distorted example]{The deep CNN estimating the VPR correctly predicts the class(id 4) with a confidence of 78.0\%.}
	\label{fig:vd2}
\end{figure}

\begin{figure}[H]
	\begin{center}
	\begin{scriptsize}
		\sffamily
		\input{img/vd3.pdf_tex}
	\end{scriptsize}
	\end{center}
	\caption[Visually distorted example]{The deep CNN estimating the VPR incorrectly predicts the class(id 2) with a confidence of 58.9\%. The correct class has id 4.}
	\label{fig:vd3}
\end{figure}

\section{Feature maps}
\label{sec:featuremaps-appendix}
The following feature maps are from the image in figure~\ref{fig:hp1}.

\begin{figure}[H]
	\begin{center}
	\begin{scriptsize}
		\sffamily
		\input{img/hardprediction.pdf_tex}
	\end{scriptsize}
	\end{center}
	\caption[Feature maps image]{The green square marks the correct class.}
	\label{fig:hp1}
\end{figure}

\begin{figure}[H]
	\begin{center}
	\begin{scriptsize}
		\sffamily
		\input{img/featuremapsvpr2.pdf_tex}
	\end{scriptsize}
	\end{center}
	\caption[Feature maps]{The feature maps highlights the position of the target behind the weapon overlay.}
\end{figure}

\begin{figure}[H]
	\begin{center}
	\begin{scriptsize}
		\sffamily
		\input{img/featuremapsar2.pdf_tex}
	\end{scriptsize}
	\end{center}
	\caption[Feature maps]{The feature maps vaguely highlights the position of the target behind the weapon overlay.}
\end{figure}


\section{Figures}

\begin{figure}[H]
	\begin{scriptsize}
		\sffamily
		\def\svgwidth{\textwidth}
		\input{img/grid.pdf_tex}
	\end{scriptsize}
	\caption{Ids of the 3, 3, 3 partitioning scheme}
	\label{fig:ids}
\end{figure}

\section{Neuroevolution graphs}
\label{sec:neuroevolution-graphs}
%!TEX root = ../../preamble.tex

\pgfplotstableread{data/neat/ang-mean.dat}{\neatAngularMean}
\pgfplotstableread{data/neat/ang-recoil-mean.dat}{\neatAngularRecoilMean}
\pgfplotstableread{data/neat/vpr-mean.dat}{\neatVPRMean}
\pgfplotstableread{data/neat/vpr-recoil-mean.dat}{\neatVPRRecoilMean}

\begin{figure}[H]
\begin{tikzpicture}[scale=1]
	\begin{axis}[
			title=Averaged aiming fitness over generations,
			xlabel=Generations,
			ylabel=Aiming fitness,
			ymax = 75,
			xmin = -10,
			xmax = 273,
			restrict x to domain=0:263,
			xticklabel style={rotate=30},
			minor tick num=1,
			legend pos=south east,
			transpose legend,
			legend columns=2,
			legend style={
				/tikz/column 2/.style={
					column sep=10pt,
				}
            },
		]
		\addplot+ [cred, mark=none] table [x={Generation}, y={AimingFitness}] {\neatAngularMean};
		\addlegendentry{AR - recoil}
		\addplot+ [corange, mark=none] table [x={Generation}, y={AimingFitness}] {\neatAngularRecoilMean};
		\addlegendentry{AR + recoil}
		\addplot+ [cblue, mark=none] table [x={Generation}, y={AimingFitness}] {\neatVPRMean};
		\addlegendentry{VPR - recoil}
		\addplot+ [cgreen, mark=none] table [x={Generation}, y={AimingFitness}] {\neatVPRRecoilMean};
		\addlegendentry{VPR + recoil}
	\end{axis}
\end{tikzpicture}
\caption[Averaged NEAT aiming fitness]{Training the agent}
\end{figure}
 %DONE
%!TEX root = ../../preamble.tex

\pgfplotstableread{data/neat/ang-mean.dat}{\neatAngularMean}
\pgfplotstableread{data/neat/ang-recoil-mean.dat}{\neatAngularRecoilMean}
\pgfplotstableread{data/neat/vpr-mean.dat}{\neatVPRMean}
\pgfplotstableread{data/neat/vpr-recoil-mean.dat}{\neatVPRRecoilMean}

\begin{figure}[H]
\begin{tikzpicture}[scale=1]
	\begin{axis}[
			title=Averaged shooting fitness over generations,
			xlabel=Generations,
			ylabel=Shooting fitness,
			ymin = 0,
			ymax = 450,
			xmin = -10,
			xmax = 273,
			restrict x to domain=0:263,
			xticklabel style={rotate=30},
			minor x tick num=1,
			legend pos=north west,
			transpose legend,
			legend columns=2,
			legend style={
				/tikz/column 2/.style={
					column sep=10pt,
				}
            },
		]
		\addplot+ [cred, mark=none] table [x={Generation}, y={ShootingFitness}] {\neatAngularMean};
		\addlegendentry{AR - recoil}
		\addplot+ [corange, mark=none] table [x={Generation}, y={ShootingFitness}] {\neatAngularRecoilMean};
		\addlegendentry{AR + recoil}
		\addplot+ [cblue, mark=none] table [x={Generation}, y={ShootingFitness}] {\neatVPRMean};
		\addlegendentry{VPR - recoil}
		\addplot+ [cgreen, mark=none] table [x={Generation}, y={ShootingFitness}] {\neatVPRRecoilMean};
		\addlegendentry{VPR + recoil}
	\end{axis}
\end{tikzpicture}
\caption[Averaged NEAT shooting fitness]{Training the agent}
\end{figure} %DONE

%!TEX root = ../../preamble.tex

\pgfplotstableread{data/neat/ang-mean.dat}{\neatAngularMean}

\begin{figure}[H]
\begin{tikzpicture}[scale=1]
	\begin{axis}[
			title=Fitness over generations,
			xlabel=Generations,
			ylabel=Total fitness,
			ymin = 0,
			ymax = 500,
			xmin = -16,
			xmax = 416,
			restrict x to domain=0:400,
			xticklabel style={rotate=30},
			minor x tick num=1,
			legend pos=north west,
		]
		\addplot+ [bluered-mix, mark=none] table [x={Generation}, y={Fitness}] {\neatAngularMean};
		\addlegendentry{Overall fitness}
		\addplot+ [cred, mark=none] table [x={Generation}, y={AimingFitness}] {\neatAngularMean};
		\addlegendentry{Aiming fitness}
		\addplot+ [cblue, mark=none] table [x={Generation}, y={ShootingFitness}] {\neatAngularMean};
		\addlegendentry{Shooting fitness}
	\end{axis}
\end{tikzpicture}
\caption{The total shooting and aiming fitness averaged without recoil for the AR. Each graph is an average of 3 runs.}
\end{figure} %DONE
%!TEX root = ../../preamble.tex

\pgfplotstableread{data/neat/ang-recoil-mean.dat}{\neatAngularRecoilMean}

\begin{figure}[H]
\begin{tikzpicture}[scale=1]
	\begin{axis}[
			title=Fitness over generations,
			xlabel=Generations,
			ylabel=Total fitness,
			ymin = 0,
			ymax = 200,
			xmin = -15,
			xmax = 391,
			restrict x to domain=0:376,
			xticklabel style={rotate=30},
			minor x tick num=1,
			legend pos=north west,
		]
		\addplot+ [bluered-mix, mark=none] table [x={Generation}, y={Fitness}] {\neatAngularRecoilMean};
		\addlegendentry{Overall fitness}
		\addplot+ [cred, mark=none] table [x={Generation}, y={AimingFitness}] {\neatAngularRecoilMean};
		\addlegendentry{Aiming fitness}
		\addplot+ [cblue, mark=none] table [x={Generation}, y={ShootingFitness}] {\neatAngularRecoilMean};
		\addlegendentry{Shooting fitness}
	\end{axis}
\end{tikzpicture}
\caption{The total shooting and aiming fitness averaged with recoil. Each graph is an average of 3 runs.}
\end{figure} %DONE
%!TEX root = ../../preamble.tex

\pgfplotstableread{data/neat/vpr-mean.dat}{\neatVPRMean}

\begin{figure}[H]
\begin{tikzpicture}[scale=1]
	\begin{axis}[
			title=Fitness over generations,
			xlabel=Generations,
			ylabel=Fitness,
			ymin = 0,
			xmin = -10,
			xmax = 225,
			restrict x to domain=0:215,
			xticklabel style={rotate=30},
			minor x tick num=1,
			legend pos=north west,
		]
		\addplot+ [bluered-mix, mark=none] table [x={Generation}, y={Fitness}] {\neatVPRMean};
		\addlegendentry{Overall fitness}
		\addplot+ [cred, mark=none] table [x={Generation}, y={AimingFitness}] {\neatVPRMean};
		\addlegendentry{Aiming fitness}
		\addplot+ [cblue, mark=none] table [x={Generation}, y={ShootingFitness}] {\neatVPRMean};
		\addlegendentry{Shooting fitness}
	\end{axis}
\end{tikzpicture}
\caption{Training the VPR agent without recoil}
\end{figure} %DONE
%!TEX root = ../../preamble.tex

\pgfplotstableread{data/neat/vpr-recoil-mean.dat}{\neatVPRRecoilMean}

\begin{figure}[H]
\begin{tikzpicture}[scale=1]
	\begin{axis}[
			title=Fitness over generations,
			xlabel=Generations,
			ylabel=Fitness,
			ymin = 0,
			ymax = 100,
			xmin = -10,
			xmax = 276,
			restrict x to domain=0:266,
			xticklabel style={rotate=30},
			minor x tick num=1,
			legend pos=north west,
		]
		\addplot+ [bluered-mix, mark=none] table [x={Generation}, y={Fitness}] {\neatVPRRecoilMean};
		\addlegendentry{Overall fitness}
		\addplot+ [cred, mark=none] table [x={Generation}, y={AimingFitness}] {\neatVPRRecoilMean};
		\addlegendentry{Aiming fitness}
		\addplot+ [cblue, mark=none] table [x={Generation}, y={ShootingFitness}] {\neatVPRRecoilMean};
		\addlegendentry{Shooting fitness}
	\end{axis}
\end{tikzpicture}
\caption{The total shooting and aiming fitness averaged with recoil for the VPR. Each graph is an average of 3 runs.}
\label{fig:neat-averaged-vpr-recoil-fitness}
\end{figure} %DONE

%!TEX root = ../../preamble.tex

\pgfplotstableread{data/neat/ang-mean.dat}{\neatAngularMean}

\begin{figure}[H]
\begin{tikzpicture}[scale=1]
	\begin{axis}[
			title=Fitness over generations,
			xlabel=Generations,
			ylabel=Total fitness,
			ymin = 0,
			ymax = 500,
			xmin = -16,
			xmax = 416,
			restrict x to domain=0:400,
			xticklabel style={rotate=30},
			minor x tick num=1,
			legend pos=north west,
		]
		\addplot+ [bluered-mix, mark=none] table [x={Generation}, y={Fitness}] {\neatAngularMean};
		\addlegendentry{Overall fitness}
		\addplot+ [cred, mark=none] table [x={Generation}, y={AimingFitness}] {\neatAngularMean};
		\addlegendentry{Aiming fitness}
		\addplot+ [cblue, mark=none] table [x={Generation}, y={ShootingFitness}] {\neatAngularMean};
		\addlegendentry{Shooting fitness}
	\end{axis}
\end{tikzpicture}
\caption{The total shooting and aiming fitness averaged without recoil for the AR. Each graph is an average of 3 runs.}
\end{figure} %DONE
%!TEX root = ../../preamble.tex

\pgfplotstableread{data/neat/ang-recoil-mean.dat}{\neatAngularRecoilMean}

\begin{figure}[H]
\begin{tikzpicture}[scale=1]
	\begin{axis}[
			title=Fitness over generations,
			xlabel=Generations,
			ylabel=Total fitness,
			ymin = 0,
			ymax = 200,
			xmin = -15,
			xmax = 391,
			restrict x to domain=0:376,
			xticklabel style={rotate=30},
			minor x tick num=1,
			legend pos=north west,
		]
		\addplot+ [bluered-mix, mark=none] table [x={Generation}, y={Fitness}] {\neatAngularRecoilMean};
		\addlegendentry{Overall fitness}
		\addplot+ [cred, mark=none] table [x={Generation}, y={AimingFitness}] {\neatAngularRecoilMean};
		\addlegendentry{Aiming fitness}
		\addplot+ [cblue, mark=none] table [x={Generation}, y={ShootingFitness}] {\neatAngularRecoilMean};
		\addlegendentry{Shooting fitness}
	\end{axis}
\end{tikzpicture}
\caption{The total shooting and aiming fitness averaged with recoil. Each graph is an average of 3 runs.}
\end{figure} %DONE
%!TEX root = ../../preamble.tex

\pgfplotstableread{data/neat/vpr-mean.dat}{\neatVPRMean}

\begin{figure}[H]
\begin{tikzpicture}[scale=1]
	\begin{axis}[
			title=Fitness over generations,
			xlabel=Generations,
			ylabel=Fitness,
			ymin = 0,
			xmin = -10,
			xmax = 225,
			restrict x to domain=0:215,
			xticklabel style={rotate=30},
			minor x tick num=1,
			legend pos=north west,
		]
		\addplot+ [bluered-mix, mark=none] table [x={Generation}, y={Fitness}] {\neatVPRMean};
		\addlegendentry{Overall fitness}
		\addplot+ [cred, mark=none] table [x={Generation}, y={AimingFitness}] {\neatVPRMean};
		\addlegendentry{Aiming fitness}
		\addplot+ [cblue, mark=none] table [x={Generation}, y={ShootingFitness}] {\neatVPRMean};
		\addlegendentry{Shooting fitness}
	\end{axis}
\end{tikzpicture}
\caption{Training the VPR agent without recoil}
\end{figure} %DONE
%!TEX root = ../../preamble.tex

\pgfplotstableread{data/neat/vpr-recoil-mean.dat}{\neatVPRRecoilMean}

\begin{figure}[H]
\begin{tikzpicture}[scale=1]
	\begin{axis}[
			title=Fitness over generations,
			xlabel=Generations,
			ylabel=Fitness,
			ymin = 0,
			ymax = 100,
			xmin = -10,
			xmax = 276,
			restrict x to domain=0:266,
			xticklabel style={rotate=30},
			minor x tick num=1,
			legend pos=north west,
		]
		\addplot+ [bluered-mix, mark=none] table [x={Generation}, y={Fitness}] {\neatVPRRecoilMean};
		\addlegendentry{Overall fitness}
		\addplot+ [cred, mark=none] table [x={Generation}, y={AimingFitness}] {\neatVPRRecoilMean};
		\addlegendentry{Aiming fitness}
		\addplot+ [cblue, mark=none] table [x={Generation}, y={ShootingFitness}] {\neatVPRRecoilMean};
		\addlegendentry{Shooting fitness}
	\end{axis}
\end{tikzpicture}
\caption{The total shooting and aiming fitness averaged with recoil for the VPR. Each graph is an average of 3 runs.}
\label{fig:neat-averaged-vpr-recoil-fitness}
\end{figure} %DONE

%!TEX root = ../../preamble.tex

\hspace{-8mm}
\begin{figure}[H]
\begin{center}
\begin{tikzpicture}
\begin{axis}[
	width=0.55\textwidth,
	height=7cm,
	enlarge x limits={abs=.425},
	ymin=0,
	ymax=0.6,
	axis y line*=none,
    axis x line*=none,
    ybar = 15pt,
	legend style={at={(0.5,-0.175)},anchor=north},
    xticklabels={
    		{AR},
		{VPR},
	},
	xtick=data,
	minor y tick num = 1,
    nodes near coords,
	every node near coord/.append style={
		/pgf/number format/fixed,
		/pgf/number format/fixed zerofill,
		/pgf/number format/precision=3
	},
]

\addplot [draw=cred, fill=cred!70] coordinates {
	(1, 0.5) %
	(0, 0.066) %
};
\addplot [draw=cgreen, fill=cgreen!70] coordinates {
	(1, 0.433) %
	(0, 0.466) %
};

\legend{Without recoil,With recoil}
\end{axis}

\end{tikzpicture}
\end{center}
\caption[Averaged reloads with full magazine per generation]{Training the agent}
\end{figure}


% Angular no recoil			10										0.5
% Angular recoil				8.66666666667					0.4333333
% VPR no recoil				1.33333333333					0.0666666
% VPR recoil					9.33333333333					0.4666666




%!TEX root = ../../preamble.tex

\pgfplotstableread{data/neat/ang-mean.dat}{\neatAngularMean}
\pgfplotstableread{data/neat/ang-recoil-mean.dat}{\neatAngularRecoilMean}
\pgfplotstableread{data/neat/vpr-mean.dat}{\neatVPRMean}
\pgfplotstableread{data/neat/vpr-recoil-mean.dat}{\neatVPRRecoilMean}

\begin{figure}[H]
\begin{tikzpicture}[scale=1]
	\begin{axis}[
			title=Averaged missed shots per generation,
			xlabel=Generations,
			ylabel=Aiming fitness,
			ymin = 0,
			ymax = 50,
			xmin = -10,
			xmax = 273,
			restrict x to domain=0:263,
			xticklabel style={rotate=30},
			minor tick num=1,
			legend pos=north west,
			transpose legend,
			legend columns=2,
			legend style={
				/tikz/column 2/.style={
					column sep=10pt,
				}
            },
		]
		\addplot+ [cred, mark=none] table [x={Generation}, y expr = {\thisrowno{5}/20}] {\neatAngularMean};
		\addlegendentry{AR - recoil}
		\addplot+ [corange, mark=none] table [x={Generation}, y expr = {\thisrowno{5}/20}] {\neatAngularRecoilMean};
		\addlegendentry{AR + recoil}
		\addplot+ [cblue, mark=none] table [x={Generation}, y expr = {\thisrowno{5}/20}] {\neatVPRMean};
		\addlegendentry{VPR - recoil}
		\addplot+ [cgreen, mark=none] table [x={Generation}, y expr = {\thisrowno{5}/20}] {\neatVPRRecoilMean};
		\addlegendentry{VPR + recoil}
	\end{axis}
\end{tikzpicture}
\caption[Averaged missed shots per generation]{The number of missed shots are averaged from 3 runs.}
\end{figure}


\section{Deeper convolutional neural networks}
\label{sec:deeper-cnns}
The deeper CNN estimating the AR was trained in the same way as the networks described in section~\ref{sec:approach}, except that the batch size was set to 5 due to memory implications. The deeper network has an additional convolutional layer and max pooling layer as well as 5 fully connected layers with 1000 neurons in each instead of 3 layer with 250. This sums up to 16 layers. The graphs show that the deeper network approximately halves the angular error of the deep network.

%!TEX root = ../preamble.tex

\pgfplotstableread{data/angular-light-deep-every-20.dat}{\angularLightDeep}
\pgfplotstableread{data/ultradeep.dat}{\angularLightUltraDeep}

\begin{figure}[H]
\begin{tikzpicture}[scale=1]
	\begin{axis}[
			height=9cm,
			title=Cost over iterations,
			xlabel=Iterations,
			ylabel=Cost,
			ymin=0,
			ymax=0.55,
			xmin = -640,
			xmax = 16640,
			restrict x to domain=0:16000,
			xticklabel style={rotate=30},
			minor tick num=1,
			legend pos=north east,
		]
		\addplot [cblue, mark=none] table [x={iteration}, y={score}] {\angularLightDeep};
		\addlegendentry{Deep network}
		\addplot [cpurple, mark=none] table [x={iteration}, y={cost}] {\angularLightUltraDeep};
		\addlegendentry{Ultra deep network}
	\end{axis}
\end{tikzpicture}
\caption{Mean squared error cost over mini-batch gradient descent iterations using AR with visual distortion}
\label{fig:score-light-angular-ultradeep}
\end{figure}
%!TEX root = ../preamble.tex

\begin{figure}[H]
\begin{tikzpicture}
\begin{axis}[
	angular-mse,
	%transpose legend,
	legend columns=2,
	legend style={
		/tikz/column 2/.style={
			column sep=10pt,
		}
	},
]

\addplot [draw=cpurple, fill=cpurple!70] coordinates {
	(0.086393, 2) % Correct value
	(0.097324, 1) % Correct value
	(0.120908, 0) % Correct value
};
\addlegendentry{Deeper}
\addplot [draw=cblue, fill=cblue!70] coordinates {
	(0.161777,2) % Correct value
	(0.172215,1) % Correct value
	(0.163934,0) % Correct value
};
\addlegendentry{Deep}

%\legend{Ultra deep, Deep}
\end{axis}

\end{tikzpicture}
\caption[Mean error of the deeper network]{Angular accuracy of the deeper topology compared to the deep topology with visual distortion, evaluated on a test set.}
\label{fig:angular-mse-light-ultra-deep}
\end{figure}

%Evaluating model: models\angular\light\ultra-deep\model1.bin
%Using db table: trainingDataLightTestSet
%Mean horizontal error is: 0.086393
%Mean vertical error is: 0.097324
%Mean distance error is: 0.120908
%Accuracy is: 0.928800













































